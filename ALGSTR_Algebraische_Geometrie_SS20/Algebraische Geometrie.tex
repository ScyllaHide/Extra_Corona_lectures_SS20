\documentclass[ngerman,a4paper,order=firstname]{mathscript}
\usepackage{mathoperators}

\usepackage{xcolor}
\usepackage{todonotes}						% put todonotes but with online, otherwise latex gets broken :/
% % % local commands
\DeclareMathOperator{\Ad}{Ad}				% Adjoint
\DeclareMathOperator{\PSL}{PSL} 			% projective linear group 
\newcommand{\with}{\text{ with }}
\newcommand{\nd}{\text{ and }}
\renewcommand{\rhd}{\triangleright}
\renewcommand{\lhd}{\triangleleft} 			% normal subgroups
\DeclareMathOperator{\Set}{Set}				% Category of sets
\DeclareMathOperator{\Vect}{Vect}			% Category of vector spaces
\DeclareMathOperator{\Grp}{Grp}				% Category of groups
\DeclareMathOperator{\Mod}{Mod}				% Cat of moduls
\DeclareMathOperator{\Ann}{Ann}				% annihilator
\newcommand{\Circlearrowleft}{\rotatebox{180}{$\circlearrowright$}}
\DeclareMathOperator{\Cl}{Cl}				% conjugation class of something.
\DeclareMathOperator{\Iso}{Iso}

% % % % Model theory
\newcommand{\LL}{\mathcal L}
\newcommand{\RR}{\mathcal R}
\newcommand{\FF}{\mathcal F}
\newcommand{\MM}{\mathcal M}
\newcommand{\M}{\mathbb M} 						% class of \LL structures
\newcommand{\NN}{\mathcal N}
\newcommand{\CC}{\mathcal C}
\newcommand{\arity}{\lambda}
\newcommand{\discup}{\overset{+}{\cup}}			% disjoint union
%\newcommand{\satisfy}{\vDash}					% M \satisfy \phi(a)
\newcommand{\satisfy}{\models}					% M \satisfy \phi(a)
\newcommand{\lsatisfy}{\mathbin{\rotatebox[origin=c]{180}{$\models$}}}					% \phi(a) \satisfy M
\DeclareMathOperator{\Diag}{Diag}				% atomic diagrams
\newcommand{\elem}{\prec}						% expansion is elementary

% % % % better \le and \ge
\renewcommand{\ge}{\geqslant}
\renewcommand{\le}{\leqslant}
\renewcommand{\iff}{\Longleftrightarrow}

\DeclareMathOperator{\group}{group}
\DeclareMathOperator{\ring}{ring}
\DeclareMathOperator{\el}{el}
\DeclareMathOperator{\theory}{Th}

% % % % color note stuff
\newcommand{\marganote}[1]{\textcolor{gray}{#1}}
\newcommand{\fehmnote}[1]{\textcolor{red}{#1}}
\newcommand{\pascalnote}[1]{\textcolor{blue}{#1}}

% % % % % % % mark stuff
\newcommand{\ital}[1]{\textit{#1}}

% get this stupid arrows:
%\usepackage{mathabx,graphicx}  % ---> add to mathoperators
%\def\Circlearrowleft{\ensuremath{%
%		\rotatebox[origin=c]{180}{$\circlearrowleft$}}}
%\def\Circlearrowright{\ensuremath{%
%		\rotatebox[origin=c]{180}{$\circlearrowright$}}}
%\def\CircleArrowleft{\ensuremath{%
%		\reflectbox{\rotatebox[origin=c]{180}{$\circlearrowleft$}}}}
%\def\CircleArrowright{\ensuremath{%
%		\reflectbox{\rotatebox[origin=c]{180}{$\circlearrowright$}}}}
%\begin{document}
%	\Huge
%	$\circlearrowleft \circlearrowright $
%	
%	$\Circlearrowleft \Circlearrowright $
%	
%	$\CircleArrowleft \CircleArrowright $

% % % local packages
%\usepackage{braids}

\newlist{remarkenum}{enumerate}{1}
\setlist[remarkenum]{label=(\alph*),ref=\theremark~(\alph*)}
\crefalias{remarkenumi}{remark}

\newlist{propenum}{enumerate}{1}
\setlist[propenum]{label=(\alph*),ref=\theproposition~(\alph*)}
\crefalias{propenumi}{proposition}

\newlist{expenum}{enumerate}{1}
\setlist[expenum]{label=(\alph*),ref=\theexample~(\alph*)}
\crefalias{expenumi}{example}

\newlist{lemmaenum}{enumerate}{1}
\setlist[lemmaenum]{label=(\alph*),ref=\thelemma~(\alph*)}
\crefalias{lemmaenumi}{lemma}

\newlist{defenum}{enumerate}{1}
\setlist[defenum]{label=(\roman*),ref=\thedefinition~(\roman*)}
\crefalias{defenumi}{definition}

\title{\textbf{ALGSTR - Algebraische Geometrie}}
\author{Dozent: Prof. Dr. \person{Arno Fehm}}

\begin{document}
\pagenumbering{roman}
\pagestyle{plain}

\maketitle

\hypertarget{tocpage}{}
\tableofcontents
\bookmark[dest=tocpage,level=1]{Inhaltsverzeichnis}

\pagebreak
\pagenumbering{arabic}
\pagestyle{fancy}
\setcounter{chapter}{0}
\chapter*{Vorwort}
Dieser Kurs findet im Rahmen des wissenschaftlichen Arbeiten im Master statt. Und im Zuge der Corona Sache auch nur als Reading-Kurs mit Chats im Matrix System. Dazu werden Fragen zum Vorlesungsskript, Übungsaufgaben besprochen. Angedacht ist eigentlich ein Seminarteil, der bisher noch ungeregelt bleibt.

Ja, dann viel Spass bei \ital{Modelltheorie bewerteter Körper}
\chapter{Introduction: Ebene algebraische Kurven}
\input{./TeX_files/chap0}
\chapter{Grundlagen Algebra, Topologie}
\paragraph*{Marker Exercise 1.4.13}
Wie bekommen ein Problem mit der Charakteristik von $\Z_p$ für $p = 2$ und dem Polynom bzw der Darstellung die gewählt wird für die ganze p-adische Zahl in $\Z_2$. Somit ist \person{Hensel}'s Lemma nicht mehr anwendbar und damit wird $\Z_2$ nicht definiertbar in $\Q_2$. Das kann repariert werden, wenn wir die Darstellung ändern und statt $X^2 \nd a^2$ einfach $X^3 \nd a^3$ setzen. Dann muss wiederrum die Rückrichtung angepasst werden? \todo[inline]{How to finish this?! How can I repair the $\Longleftarrow$?}

\paragraph*{Marker Exercise 2.5.6 (Skolem's Paradox)}
Aus den Axiomen von ZFC folgt die Aussage "Es gibt ein x, das überabzählbar ist". Es gibt also in so einem abzählbaren Modell M ein x, so dass $M\models$ "$x$ ist überabzählbar". Schon ein wenig seltsam.
Da ich gleich in die nächste Vorlesung muss, werde ich das kurz selbst auflösen:
Der Punkt ist einfach, dass überabzählbar=nicht abzählbar über die Nichtexistenz einer Bijektion der Menge mit $\aleph_0$ definiert ist. Im abzählbaren Modell existiert zwar dieses x, aber die Menge, die den Graph einer Bijektion mit $\aleph_0$ beschreiben würde, ist nicht in M. Aus Sicht von M ist die Menge daher nicht abzählbar, auch wenn sie es von "außen" betrachtet vielleicht ist. 
Gut, gegebenenfalls können wir das nächsten Freitag weiterdiskutieren.

\paragraph*{Marker Exercise 2.5.14 Verbindung zu Corollary 3.5}
Nicht vollständig ist am einfachsten:
Also nicht vollständig klar, weil es endliche Modelle hat, und dann Formeln die was über die Anzahl der Elemente sagen manchmal erfüllt sind und manchmal nicht
AF
Genau, $Z/2Z \nd (Z/2Z)^2$ sind beides Modelle der Theorie. Elementar äquivalente endliche Strukturen sind isomorph, die beiden sind es aber offensichtlich nicht, somit ist die Theorie nicht vollständig.
Allen klar? Schön, dann zur Kategorizität. Da kann man sich es relativ einfach machen, wenn man erkennt, welches Gebiet der Algebra hierfür zuständig ist.
MK
Abelsche Gruppen, in denen jedes Element Ordnung zwei hat sind immer Z/2Z Vektorräume. Zwei Z/2Z Vektorräume sind isomorph wenn sie dieselbe Dimension haben. und weil Z/2Z endlich ist bedeutet das bei unendlich dimensionalen Vektorräumen, dass die Kardinalität gleich der Dimension ist
AF
Und gut, man braucht ein wenig Kardinalzahlarithmetik.
Genau. Vektorräume über dem Körper mit zwei Elementen, und der Rest ist Lineare Algebra.
Hm, da wäre ich mir nicht sicher. Vektorraum der Dimension kappa heißt ja eher $\sum_{\alpha<\kappa}Z/2Z$. Also direkte Summe anstatt direktes Produkt.
Wir haben eine Basis der Kardinalität $\kappa$, und jedes Element ist eine endlich Linearkombination der Basiselemente, sprich eine Summe von endlich vielen Elementen der Basis.
Man muss sich dann nur noch überlegen, dass es nur $\kappa$ viele solche endlichen Summen gibt. 
Da jede solche Summe aber höchstens abzählbar viele Elemente der Basis verwendet, bekommt man eine obere Schranke von $\aleph_0*\kappa=\kappa.$
Nein, das war Quatsch.
Man überlegt sich, dass es $\kappa^n=\kappa$ viele endliche Folgen  der Länge n von Basiselementen gibt. Über alle n erhält man dann $\aleph_0*\kappa=\kappa$ viele endliche Folgen von Basiselementen. Bei Marker Lemma A.14 und Corollary A.15(i).
Ist halt die Frage was rauskommt wenn man unendlich viele einsen aufsummiert. 0 oder 1?

Also nach Definition einer Basis eines Vektorraumes ist jedes Element eine endliche Linearkombination der Basiselemente.

Vielleicht denken Sie an so was wie das direkte Produkt $\prod_{i<\omega}Z/2Z$. Aber da ist das naheliegenden System $(\delta_{ij})_i$ eben keine Basis!
Es gibt dann noch den etwas seltsamen Zusatz "Find $T'\supseteq T$ a complete theory with the same infinite models as T."
Was machen wir da?
MK
noch die Formeln
$\exists x_1\dots x_n \mit x_1\neq x_2 \nd x_1 \neq x_3 \dots$
für jedes n dazunehmen, um endliche Modelle auszuschließen
AF
Und warum vollständig?
Margarete Ketelsen
Wegen dem Korollar
EJ
Vaught's Test?
AF
Ja genau, $\kappa$-kategorisch für eine (hier sogar jede) unendliche Kardinalzahl und keine endlichen Modelle impliziert vollständig. Aha, so heißt das Korollar 3.5 bei Marker. Schön, damit hätten wir diese Aufgabe.
\chapter{Affine Varietäten}
\section{Review of Graph Theory}
\begin{definition}[Graph]
	content...
\end{definition}
\begin{definition}[Adjancent, neighbour, degree]
	content...
\end{definition}
\begin{definition}[Graph isomorphism]
	content...
\end{definition}
\begin{definition}[Path, Cycles]
	content...
\end{definition}
\begin{definition}[Tree, forest]
	content...
\end{definition}
\begin{proposition}[Characterising Tree]
	content...
\end{proposition}
\begin{definition}[Spanning Tree]
	content...
\end{definition}

\section{Cayley Graph}
\begin{definition}[Cayley Graph]
	content...
\end{definition}
\begin{example}[Cayley graphs]
	content...
\end{example}
\begin{remark}[Elementary properties of Cayley graphs]
	content...
\end{remark}

\section{Cayley Grap of free groups}
\begin{theorem}[Cayley graph of free groups]
	content...
\end{theorem}
\begin{example}[Non-free groups with Cayley trees]
	content...
\end{example}
\begin{theorem}[Cayley trees and free groups]
	content...
\end{theorem}
\subsection{Free groups and reduced words}
\begin{definition}[Reduced word]
	content...
\end{definition}
\begin{proposition}[Free groups via reduced words]
	content...
\end{proposition}
\begin{conclusion}[Noral form of free groups]
	content...
\end{conclusion}
\begin{conclusion}[Word problem for free groups]
	content...
\end{conclusion}
\subsection{Free groups $\to$ trees}
\begin{conclusion}[title]
	Let $\omega \in (S \cup \hat{S})$ reduced and $\omega \neq \epsilon$. Then is $\omega \neq \one$ in $F(s)$.
\end{conclusion}
\begin{proof}
	Idea of the proof. Let $S = \set{a,b}$ and $\omega = a^2ba^{-1}b^{-1}$. We construct a homomorphisms
	\begin{align*}
		\phi\colon F(s) \to \Sym_n \with \phi(\omega) \neq \one_{\Sym_n}.\\
		\hom_{\Group}(F(s), \Sym_n) = \ker \circ \phi(S,\Sym_n)
	\end{align*}
	\todo{add tikz graph? - yes}
\end{proof}

\section{Exercises}
\chapter{whatever}
%\input{./TeX_files/chap3}
%\chapter{whatever}
%\input{./TeX_files/chap4}
%\chapter{whatever}
%\input{./TeX_files/chap5}

\part*{Anhang}
\addcontentsline{toc}{part}{Anhang}
\appendix

\nocite{*}
%\bibliography{literatur}
\bibliographystyle{acm}

%\printglossary[type=\acronymtype]

\printindex

\end{document}
