%packages
\usepackage[ngerman]{babel}
\usepackage{amsfonts, amssymb, amsmath, amsthm}
\usepackage{aligned-overset}
\usepackage{mleftright}                   % for \DeclarePairedDelimiter
% with \overset{\text{Definition}}&{=} instead of &\stackrel{\text{Definition}}=
\usepackage{enumerate}
%\usepackage{minipage}                     % part pages
\usepackage{enumitem}
\usepackage{csquotes}
\usepackage{hyperref}
\usepackage{bbm}                          % mathbb 1
\usepackage{mathrsfs}                     % Category-like letters for MINT with \mathscr
\usepackage{todonotes}                    % put notes on the margins
%\usepackage{pdfcomment}                   % as alternative, if todonotes conflicts with tikz!


\usepackage{fdsymbol}                         % for cloverleaf symbol
\usepackage[left=1.5cm, right=3.5cm, top=2cm]{geometry}                     % set the paper size  

% tikz stuff
\usepackage{tikz, amsmath}
	\usepackage{tikz-qtree}
	\usetikzlibrary{cd}
	\usetikzlibrary{arrows}
	\usetikzlibrary{automata}
	\usetikzlibrary{babel}
	\usetikzlibrary{calc}
	\usetikzlibrary{fit}
	\usetikzlibrary{matrix}
	\usetikzlibrary{positioning}
	\usetikzlibrary{shapes.geometric}
	\usetikzlibrary{arrows.meta,bending}

% environments

%\newtheorem {theorem} {Theorem} [section]

% das "`[theorem] "` bewirkt, dass es eine durchgehende
% Nummerierung dieser environments gibt und nicht jede
% Art seine eigene Nummerierung hat

% Für Lemmas dieses environment:
%\newtheorem {lemma} [theorem] {Lemma}
% Für Folgerungen/ Korollare dieses environment 
%\newtheorem {corollary}  {Korollar}

% Definitionen sehen dann anders aus, siehe 
% http://tex.stackexchange.com/questions/38260/non-italic-text-in-theorems-definitions-examples
%\theoremstyle {definition}
% Für Definitionen dieses environment
%\newtheorem {definition}  {Definition}

% Für Bemerkungen dieses environment
%\theoremstyle {remark}
%\newtheorem* {note} {Bemerkung}

% Beispiele kriegen eine andere Formatierung %ohne Nummerierung falls mit *:
%\newtheorem{example}  {Beispiel}
%\newtheorem{exercise}{Übung}

% ohne Nummerierung
\newtheorem* {theorem} {Theorem}

% das "`[theorem] "` bewirkt, dass es eine durchgehende
% Nummerierung dieser environments gibt und nicht jede
% Art seine eigene Nummerierung hat

% Für Lemmas dieses environment:
\newtheorem* {lemma} {Lemma}
\newtheorem* {satz} {Satz}
% Für Folgerungen/ Korollare dieses environment 
\newtheorem* {corollary} {Korollar}

% Definitionen sehen dann anders aus, siehe 
% http://tex.stackexchange.com/questions/38260/non-italic-text-in-theorems-definitions-examples
\theoremstyle {definition}
% Für Definitionen dieses environment
\newtheorem* {definition} {Definition}

% Für Bemerkungen dieses environment
\theoremstyle {remark}
\newtheorem* {remark} {Bemerkung}

\theoremstyle {remark}
\newtheorem* {problem} {Problem}

\theoremstyle {remark}
\newtheorem* {interpret} {Interpretation}

\theoremstyle {remark}
\newtheorem* {notation} {Notation}

\theoremstyle {remark}
\newtheorem* {summary} {Zusammenfassung}

\theoremstyle {remark}
\newtheorem* {overview} {Überblick}

% Beispiele kriegen eine andere Formatierung %ohne Nummerierung falls mit *:
\newtheorem*{example} {Beispiel}
\newtheorem*{exercise} {Übung}

%transpose
\newcommand\transpose[1]{\ensuremath{#1^\mathsf{T}}}
%new macro for "equals" ^=
\newcommand\equalhat{\mathrel{\stackon[1.5pt]{=}{\stretchto{%
				\scalerel*[\widthof{=}]{\wedge}{\rule{1ex}{3ex}}}{0.5ex}}}}

%fraction with backslash
\newcommand\bsfrac[2]{%
	\scalebox{-1}[1]{\nicefrac{\scalebox{-1}[1]{$#1$}}{\scalebox{-1}[1]{$#2$}}}%
}

%%%% geometry


%%%%%%%% General newcommands %%%%%%%%%%%

% Sets
%%%%%%%%%%% Old ones, to be compatible with old latex scripts %%%%%%%%%%%

\newcommand{\natur}{\mathbb{N}} % natural number Symbol
\newcommand{\whole}{\mathbb{Z}} % whole number Symbol
\newcommand{\ratio}{\mathbb{Q}} % rational number symbol
\newcommand{\real}{\mathbb{R}} % real set R
\newcommand{\comp}{\mathbb{C}} % complex set C
\newcommand{\quat}{\mathbb{H}} % quaternion
\newcommand{\field}{\mathbb{F}} % general field for the others above!

%%%% Alternative emptyset symbol %%%%%%%%%%%%%%%%%
\let\oldemptyset\emptyset
\let\emptyset\varnothing

%%%%%%%%%%% new shorter ones %%%%%%%%%%%%%%%%%

\newcommand{\N}{\mathbb{N}} 								% natural number Symbol
\newcommand{\M}{\mathbb{M}}  % Monoid
\newcommand{\G}{\mathbb{G}}  % action network
\newcommand{\Z}{\mathbb{Z}}
\renewcommand{\P}{\mathbb{P}}   % Measure or here freestyle.
\newcommand{\W}{\mathbb W}   % nice W
% whole number Symbol
\newcommand{\Q}{\mathbb{Q}} 								% rational number symbol
\newcommand{\R}{\mathbb{R}} 								% real set R
\newcommand{\Rn}{\mathbb{R}^n}
\newcommand{\Rd}{\mathbb{R}^d} 								% real set R^n for MINT
\newcommand{\Rnn}{\mathbb{R}^{n \times n}} 					% real set R^nxn
\undef{\H}
\undef{\C}
\newcommand{\C}{\mathbb{C}} 								% complex set C
\newcommand{\Cn}{\mathbb{C}^n}
\newcommand{\Cnn}{\mathbb{C}^{n \times n}} 					% complex set C^nxn
\newcommand{\ii}{\mathrm{i}\mkern1mu}    					% imaginary unit
\newcommand{\H}{\mathbb{H}} 								% quaternion
\newcommand{\F}{\mathbb{F}} 								% general field for the others above!
\newcommand{\Fp}{\F_p} 									% general field for the others above!
\renewcommand{\O}{\mathscr O}

\newcommand{\powerset}{\mathcal{P}} 						% Powerset
\newcommand{\pows}{\mathcal{P}} 							
% shoter one
%%%%%%%%%%% p-adics %%%%%%%%%%%

\newcommand{\pZ}{\mathbb{Z}_p}
\newcommand{\pQ}{\mathbb{Q}_p}
\newcommand{\pC}{\mathbb{C}_p}

%%%% Highlighting
\renewcommand{\emph}[1]{\underline{#1}}

% % % % % % % Shorter commands % % % % % % % % % % % % % % % %
\newcommand{\mal}{\cdot}                   
%\newcommand{\implies}{\Longrightarrow}        
\undef{\over}											% overrides the old Tex command \over, which are basics fractions!
\newcommand{\over}[2]{\overset{#1}{#2}} 				% shorter \overset!
\newcommand{\astar}[1]{#1^{\ast}}	                    % gesternte Sachen!
\newcommand{\aastar}[1]{#1^{\ast\ast}}                  % doppelt gesternte Sachen!
\newcommand{\uline}[1]{\underline{#1}} 					% simple underline		    

%%%%%%%%%%% Old commands, need to check, which ones are in use, if not delete!

\newcommand{\diff}{\,\mathrm{d}} 							% differential d
\renewcommand{\d}{\,\mathrm{d}} 							% shorter command for  differential d

%%%%%%%%%%%%%%%%%%%%%%%%%%%%

\newcommand{\skalar}[2]{\left\langle #1,#2\right\rangle}
\newcommand{\qraum}[2]{\sfrac{#1}{#2}}

%%% Metrics, Norms, etc %%%%

% Norms, Absolute, Scalar product
% Source: https://tex.stackexchange.com/questions/94410/easily-change-behavior-of-declarepaireddelimiter
% Example: \set{...}; \set*{...}=\set[]{...}; \set[\big]{...}, \set[\Big]{...}, ...
% binar form of this command made by https://github.com/lostindarkmath, see https://github.com/henrydatei/TUD_MATH_BA/issues/66
\NewDocumentCommand\xDeclarePairedDelimiter{mmm} % unar
{
	\NewDocumentCommand#1{som}{
		\IfNoValueTF{##2}
		{\IfBooleanTF{##1}{#2##3#3}{\mleft#2##3\mright#3}}
		{\mathopen{##2#2}##3\mathclose{##2#3}}
	}
}

\NewDocumentCommand\xDeclarePairedDelimiterBin{mmm} % binar
{
	\NewDocumentCommand#1{somm}{
		\IfNoValueTF{##2}
		{\IfBooleanTF{##1}{#2##3,##4#3}{\mleft#2##3,##4\mright#3}}
		{\mathopen{##2#2}##3,##4\mathclose{##2#3}}
	}
}

\xDeclarePairedDelimiter{\set}{\lbrace}{\rbrace}
\xDeclarePairedDelimiter{\norm}{\Vert}{\Vert}
\xDeclarePairedDelimiter{\abs}{\vert}{\vert}
\xDeclarePairedDelimiterBin{\scaProd}{\langle}{\rangle}
\xDeclarePairedDelimiter{\brackets}{(}{)}
\xDeclarePairedDelimiter{\sqbrackets}{[}{]}

%%%%%%%%%%%%%%%%% MINT %%%%%%%%%%%%%%%%%%%%%%%%%%%%%%%%%%%%%%%%%

%\newcommand{\sigmalg}{$\sigma$-Algebra}
% declares the Bourbaki’s ‘dangerous bend symbol, to highlight common Errors done with definitions, theorems, find it quite useful.
\DeclareFontFamily{U}{stixbbit}{}
\DeclareFontShape{U}{stixbbit}{m}{it}{<-> stix-mathbbit}{}
\DeclareRobustCommand{\stixdanger}{%
  {\usefont{U}{stixbbit}{m}{it}\symbol{"F6}}%
}

%%%%%%%%%%% Measure Theory %%%%%%%%%%%%%%%%%%

\newcommand{\sigA}{\mathscr{A}}         % typical Sigma A for sigma algebras
\newcommand{\sigB}{\mathscr{B}}         % typical Sigma B for sigma algebras
\newcommand{\sigG}{\mathscr{G}}         % typical Generator for sigma algebras
\newcommand{\sigO}{\mathscr{O}}         % sigma algebra of open sets
\newcommand{\probp}{\mathbb{P}}         % probability measure
\newcommand{\borel}{\mathscr{B}}        % Borel set
\newcommand{\sigF}{\mathscr{F}}         % typical sigma F for sigma algebra for stochastic lecture
\newcommand{\indi}{\mathbbm{1}}         % Indicator function

%%%%%%%%%%% Stochastic %%%%%%%%%%%%%%%%%%

% distributions
\DeclareMathOperator{\Bin}{Bin}         % Binomial-distribution
\DeclareMathOperator{\Ber}{Bernoulli}   % Bernoulli-distibution
\DeclareMathOperator{\Multi}{Multi}     % Multinomial-distribution
\DeclareMathOperator{\Hyper}{Hyper}     % hypergeometric-distribution
\DeclareMathOperator{\Pois}{Poisson}    % Poisson-distribution
\DeclareMathOperator{\Gleich}{U}        % Gleich-distribution
\DeclareMathOperator{\negBin}{negBin}   % negative Bin-distribution
\DeclareMathOperator{\Geom}{Geom}        % geometrische-distribution
\DeclareMathOperator{\Gam}{Gamma}
\DeclareMathOperator{\EXP}{Exp}
\newcommand{\Ln}[1]{\mathscr{L}^{#1}} % L spaces

% Other stuff for stochastic

%%%%%%%%%%%%%%%%%%%%%%%%%%%%%%% Math Operators %%%%%%%%%%%%%%%%%%%%%%%%%%%%%%%

%%%%%% A
\DeclareMathOperator{\Aff}{Aff} 				% Affine
\DeclareMathOperator{\Aut}{Aut} 				% Automorphism
\DeclareMathOperator{\Abb}{Abb} 				% Abbildung
\DeclareMathOperator{\add}{add}

%%%%%% B
\DeclareMathOperator{\Bil}{Bil}					% Bilinear form

%%%%%% C
\DeclareMathOperator{\cl}{cl} 					% Closure
\DeclareMathOperator{\cond}{cond} 				% ? 
\DeclareMathOperator{\curl}{curl} 				% Curl/Rotation
\DeclareMathOperator{\chara}{char} 				% characteric Ring/Field
\newcommand{\cloverleaf}{\clubsuit}             % cloverleaf symbol for the cloverleaf network
\newcommand{\compare}{\nearrow}                 % compare literature, other branch, etc

%%%%%% D
\DeclareMathOperator{\diam}{diam} 				% Diameter of a set
\undef\div
\DeclareMathOperator{\div}{div} 				% Divergence
\DeclareMathOperator{\diag}{diag} 				% Diagonal matrix
\DeclareMathOperator{\D}{d} 					% ?
\DeclareMathOperator{\Discr}{discr}

%%%%%% E
\DeclareMathOperator{\Eig}{Eig} 				% Eigenspace
\DeclareMathOperator{\End}{End} 				% Endomorphism
\DeclareMathOperator{\ext}{ext} 				% Exterior
\DeclareMathOperator{\Ext}{ext} 				% 2nd time?

%%%%%% F
\DeclareMathOperator{\Fix}{Fix} 				% Fix-points for group theory

%%%%%% G
\DeclareMathOperator{\ggT}{ggT}					% Größte geimeinsamer Teiler (gcd)
\DeclareMathOperator{\GL}{GL} 					% General Linear Group
\DeclareMathOperator{\graph}{graph} 			% Graph
\DeclareMathOperator{\grad}{grad} 				% Gradient
\DeclareMathOperator{\Gal}{Gal}
\newcommand{\Gen}[1]{\langle #1\rangle}

%%%%%% H
\DeclareMathOperator{\Hom}{Hom} 				% family of Homomorphisms
\DeclareMathOperator{\Hau}{Hau} 				% Hauptraum (generalized Eigenspace)

%%%%%% IJK
\DeclareMathOperator{\Image}{Im} 				% Image
\DeclareMathOperator{\Inn}{Inn} 				% Inner autommorphism
\DeclareMathOperator{\Int}{int} 				% 2nd time
\DeclareMathOperator{\id}{id} 					% Identity

\DeclareMathOperator{\kgV}{kgV} 				% kleinste gemeinsamer Teiler (lcd)
\DeclareMathOperator{\Ker}{Ker} 				% kernel of a morphisms

%%%%%% LMN
\DeclareMathOperator{\LC}{LC} 					% Leitkoeffizient polynome

\DeclareMathOperator{\Mat}{Mat} 				% Matrix
\DeclareMathOperator{\Measure}{\mathscr M} 
\DeclareMathOperator{\MinPol}{MinPol} 			% Minimal Polynomial
\DeclareMathOperator{\multi}{multi}             % multiplication label
\DeclareMathOperator{\Map}{Map}
\DeclareMathOperator{\ap}{ap}


%%%%%% OPQ
\DeclareMathOperator{\ord}{ord} 				% Order of a group
\DeclareMathOperator{\Orth}{O} 					% Orthogonal Group
\DeclareMathOperator{\Out}{Out} 				% Outer Automorphism
\DeclareMathOperator{\ordd}{ord}                 % ord

\DeclareMathOperator{\Landau}{\mathcal{O}} 				% Big ``O'' (Landau)
\DeclareMathOperator{\Path}{Path}

\DeclareMathOperator{\pr}{pr} 					% Projection

\DeclareMathOperator{\Quot}{Quot} 				% Quotient

%%%%%% R
\DeclareMathOperator{\rd}{rd} 					% ?
\DeclareMathOperator{\rk}{rk} 					% Rank of a matrix
\DeclareMathOperator{\rot}{rot} 				% Rotation (Curl)
\DeclareMathOperator{\rang}{rang} 				% Matrix Rank
\DeclareMathOperator{\res}{res}
\newcommand{\Relat}{\mathbb{P}}

%%%%%% S
\DeclareMathOperator{\sgn}{sgn} 				% Signum function
\DeclareMathOperator{\SL}{SL} 					% Special Linear Group
\DeclareMathOperator{\SO}{SO} 					% Special Orthogonal Group
\DeclareMathOperator{\Span}{span} 				% Span
\DeclareMathOperator{\SR}{SR} 					% Spaltenraum
\DeclareMathOperator{\Stab}{Stab} 				% Stabilizer
\DeclareMathOperator{\supp}{supp} 				% Support
\DeclareMathOperator{\Syl}{Syl} 				% Sylow Group
\DeclareMathOperator{\Sym}{Sym} 				% Symmetric Group
\DeclareMathOperator{\SU}{SU} 					% Special Unitary Group

%%%%%% T
\DeclareMathOperator{\tr}{tr} 					% Trace operator
\DeclareMathOperator{\Typ}{Typ} 				% Type
\DeclareMathOperator{\trdeg}{tr.deg}

%%%%%% UVW
\DeclareMathOperator{\UG}{UG} 					% Untergruppe (subgroup) 
\DeclareMathOperator{\Uni}{U} 					% Unitary Group
\DeclareMathOperator{\ugr}{Ugr}					%Set of Subgroups (ALGZTH)

\DeclareMathOperator{\w}{w}
\DeclareMathOperator{\WWord}{\W\ordd}
\DeclareMathOperator{\Word}{Word}
% |Word
\DeclareMathOperator{\vect}{vec}
%%%%%% XYZ
\DeclareMathOperator{\ZR}{ZR} 					% Zeilenraum
\DeclareMathOperator{\zwk}{Zwk}					%Zwischenkörper / ALGZTH
%\DeclareMathOperator{\Z}{Z} 					% ?


% WHITESPACE COMMANDS
\newcommand{\nl}{\\[\baselineskip]}      		% wordwrap with empty Zeile line after w/o underfull-hbox-warning
\newcommand{\enter}{$ $\newline}         		% wordwrap w/o condition
\newcommand\tab[1][1cm]{\hspace*{#1}}    		% practical Tabulator

% german words in align env %
\renewcommand{\mit}{\text{ mit }}        		% overrides \mit - Used to turn on math italic typestyle.
\newcommand{\bzw}{\text{ bzw. }}         		% sprachliches bzw. fГјr Numerik
\newcommand{\falls}{\text{falls }}       		% sprachliches falls fГјr Fallunterscheidungen
\newcommand{\sonst}{\text{sonst }}       		% sprachliches sonst fГјr Fallunterscheidungen
\newcommand{\und}{\text{ und }}          		% sprachliches und fГјr align-Umgebungen
\newcommand{\oder}{\text{ oder }}        		% sprachliches oder fГјr align-Umgebungen

% english words in align env %
\newcommand{\with}{\text{ with }}				% sprachliches with fГјr align-Umgebungen
\newcommand{\nd}{\text{ and }}					% sprachliches und fГјr align-Umgebungen
%\newcommand{\else}{\text{else }}				% sprachliches else fГјr Fallunterscheidungen
%\newcommand{\or}{\text{ or }}					% sprachliches or fГјr align-Umgebungen
%\newcommand{\if}{\text{if }} 					% sprachliches if fГјr Fallunterscheidungen
\newcommand{\mylinebreak}{\mbox{}}              % proper break lines after paragraphs

% overset text
\newcommand{\defeq}{\overset{\text{Def}}{=}}    % Definition over over =

\let\implies\Rightarrow

%%%%%% circle with stuff inside

\makeatletter
\newcommand{\ostar}{\mathbin{\mathpalette\make@circled\star}}
\newcommand{\make@circled}[2]{%
  \ooalign{$\m@th#1\smallbigcirc{#1}$\cr\hidewidth$\m@th#1#2$\hidewidth\cr}%
}
\newcommand{\smallbigcirc}[1]{%
  \vcenter{\hbox{\scalebox{0.77778}{$\m@th#1\bigcirc$}}}%
}